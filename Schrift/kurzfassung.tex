\begin{abstract}
Für die Realisierung Ressourcen-orientierter Kommunikation sind REST-Services oftmals das Mittel der Wahl.
Dabei kommuniziert das Frontend mit dem Backend um die, für die Visualisierung benötigen Daten, abzufragen.
Dabei ist die Datenabfrage mittels REST jedoch sehr unflexibel.
Bei komplexeren Datenstrukturen reicht eine einzelne Abfrage oftmals nicht aus um alle für den Client relevanten Daten bereitzustellen.
Weiters können Probleme auftreten die auf die Unflexibilität von REST zurückzuführen sind.
Um der Unflexibilität von REST und den auftretenden Problemen entegenzuwirken, wurde von Facebook GraphQL entwickelt.
GraphQL bietet eine weitere Möglichkeit Web-APIs zu realiseren und dabei die Probleme von REST zu minimieren.
Weiters bietet GraphQL eine flexiblere und effizientere Möglichkeit Daten abzufragen.
\newline

In dieser Arbeit werden die konzeptionellen Grundlagen von GraphQL aufgearbeitet.
Weiters wird der Entwicklungsprozess eines GraphQL-Service beschrieben.
Der aus der Entwicklung resultierende Prototyp wird mit .NET 6 und unter Zuhilfenahme des HotChocolate Frameworks umgesetzt.
Für den Datenbankzugriff wird das Entity Framework herangezogen.
Die Umsetzung beschäftigt sich zudem mit der Lösung von bekannten Problemen wie dem 1+n Problem oder dem Verhindern von Underfetching und Overfetching.
Weiters wird die Absicherung des GraphQL-Service mit Authentifizierung und Autorisierung durch unberechtigte Zugriffe von Außen behandelt.
Zudem wird die bidirektionale Kommunikation zwischen Backend und Frontend mittels Subscriptions realisert.
\end{abstract}

\begin{abstract}
For the realisation of resource-oriented communication, REST services are often the means of choice.
The frontend communicates with the backend to retrieve the data needed for the visualisation.
Data retrieval via REST is very inflexible.
With complex data structures, a single query is often not sufficient to provide all the data relevant for the client.
Furthermore, problems can occur due to the inflexibility of REST.
To counteract the inflexibility of REST and the problems that arise, Facebook developed GraphQL.
GraphQL offers another possibility to realise Web-APIs and to minimise the problems of REST.
Furthermore, GraphQL offers a more flexible and efficient way to retrieve data.
\newline

In this thesis the conceptual basics of GraphQL are presented.
Furthermore, the development process of a GraphQL-Service is described.
The resulting prototype is implemented with .NET 6 and the HotChocolate Framework.
The Entity Framework is used for database access.
The implementation also deals with the solution of known problems such as the 1+n problem or the prevention of under and overfetching.
Furthermore, the GraphQL service is secured against unauthorised access from outside with authentication and authorisation.
In addition, the bidirectional communication between backend and frontend is realised by means of subscriptions.
\end{abstract}