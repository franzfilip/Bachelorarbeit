\chapter{Einleitung}

\section{Motivation}
% bietet flexiblere möglichkeit für den client auf das backend zuzugreifen
% In modernen Anwendungen werden REST-Services oftmals für die Datenübertragung herangezogen.
% Für die Ressourcen-orientierte Kommunikation zwischen Client und Server werden oftmals REST-Services herangezogen.
Für die Realisierung Ressourcen-orientierter Kommunikation sind REST-Services oftmals das Mittel der Wahl.
Dabei kommuniziert das Frontend mit dem Backend um die, für die Visualisierung benötigen Daten, abzufragen.
% Der Client kommuniziert dabei mit dem Server über diese Schnittstelle um die für ihn benötigen Daten abzufragen.
Dabei ist die Datenabfrage mittels REST jedoch sehr unflexibel.
Bei komplexeren Datenstrukturen reicht eine einzelne Abfrage oftmals nicht aus um alle für den Client relevanten Daten bereitzustellen.
Weiters können Probleme auftreten die auf die Unflexibilität von REST zurückzuführen sind.
Um der Unflexibilität von REST und den auftretenden Problemen entegenzuwirken, wurde von Facebook GraphQL entwickelt.
% Eine weitere Möglichkeit Web-APIs zu realiseren, bietet die von Facebook entwickelte Abfragesprache GraphQL.
GraphQL bietet eine weitere Möglichkeit Web-APIs zu realiseren und dabei die Probleme von REST zu minimieren.
% GraphQL bietet Möglichkeiten die Probleme von REST zu minimieren oder zu beheben.
Weiters bietet GraphQL eine flexiblere und effizientere Möglichkeit Daten abzufragen.
\newline

In dieser Arbeit werden die konzeptionellen Grundlagen von GraphQL aufgearbeitet.
Weiters wird der Entwicklungsprozess eines GraphQL-Service beschrieben.
Der aus der Entwicklung resultierende Prototyp wird mit .NET 6 und unter Zuhilfenahme des HotChocolate Frameworks umgesetzt.
Für den Datenbankzugriff wird das Entity Framework herangezogen.
Die Umsetzung beschäftigt sich zudem mit der Lösung von bekannten Problemen wie dem 1+n Problem oder dem Verhindern von Under und Overfetching.
Weiters wird die Absicherung des GraphQL-Service mit Authentifizierung und Autorisierung durch unberechtigte Zugriffe von außen abgesichert.
Weiters wird mittels dem Prototypen eine bidirektionale Kommunikation zwischen Client und Server realisiert. 
Zudem wird die bidirektionale Kommunikation zwischen Backend und Frontend mittels Subscriptions realisert.

\section{Zielsetzung}
placeholder
\pagebreak