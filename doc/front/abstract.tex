\chapter{Abstract}

\begin{english} %switch to English language rules
For the realisation of resource-oriented communication, REST services are often the means of choice.
The frontend communicates with the backend to retrieve the data needed for the visualisation.
Data retrieval via REST is very inflexible.
With complex data structures, a single query is often not sufficient to provide all the data relevant for the client.
Furthermore, problems can occur due to the inflexibility of REST.
To counteract the inflexibility of REST and the problems that arise, Facebook developed GraphQL.
GraphQL offers an alternative option to realise Web-APIs and to minimise the problems of REST.
Furthermore, GraphQL offers a more flexible and efficient way to retrieve data.
\newline

In this thesis the conceptual basics of GraphQL are presented.
Moreover, the development process of a GraphQL-Service is described.
The resulting prototype is implemented with .NET 6 and the HotChocolate framework.
The Entity Framework is used for database access.
The implementation also deals with the solution of known problems such as the 1+n problem or the prevention of under and overfetching.
Additionally, the GraphQL service is secured against unauthorised access from outside.
Last but not least, it is shown that bidirectional communication between backend and frontend can be realised by means of subscriptions.
    
\end{english}