\chapter{Kurzfassung}
\label{ch:kurzfassung}

Für die Realisierung Ressourcen-orientierter Kommunikation sind REST-Services oftmals das Mittel der Wahl.
Dabei kommuniziert das Frontend mit dem Backend, um die für die Visualisierung benötigen Daten abzufragen.
Dabei ist die Datenabfrage mittels REST jedoch sehr unflexibel.
Bei komplexeren Datenstrukturen reicht eine einzelne Abfrage oftmals nicht aus, um alle für den Client relevanten Daten bereitzustellen.
Weiters können Probleme auftreten, die auf die Unflexibilität von REST zurückzuführen sind.
Um der Unflexibilität von REST und den auftretenden Problemen entegenzuwirken, wurde von Facebook GraphQL entwickelt.
GraphQL bietet eine alternative Möglichkeit, Web-APIs zu realiseren und dabei die Probleme von REST zu minimieren.
Weiters bietet GraphQL eine flexiblere und effizientere Möglichkeit, Daten abzufragen.
\newline

In dieser Arbeit werden die konzeptionellen Grundlagen von GraphQL aufgearbeitet.
Weiters wird der Entwicklungsprozess eines GraphQL-Service beschrieben.
Der aus der Entwicklung resultierende Prototyp wird mit .NET 6 unter Zuhilfenahme des Frameworks HotChocolate umgesetzt.
Für den Datenbankzugriff wird das Entity Framework herangezogen.
Die Umsetzung beschäftigt sich zudem mit der Lösung von bekannten Problemen bei der Implementierung von GraphQL-Services wie dem \glqq1+n Problem\grqq{} oder dem Verhindern von Underfetching und Overfetching.
Weiters wird die Absicherung vom GraphQL-Service behandelt.
Zudem wird auf die bidirektionale Kommunikation zwischen Backend und Frontend mittels Subscriptions eingegangen.
